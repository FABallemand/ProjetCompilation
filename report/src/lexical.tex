\section{Analyse Lexicale}

\paragraph{}
La première étape de la compilation consiste à analyser les unités lexicales contenues dans un programme, c'est à dire découper le programme en blocs de taille la plus petite possible selon la syntaxe du langage de programmation.

\paragraph{}
L'analyse lexicale est réalisée à l'aide de \textsf{Flex}. Cet outil permet de définir des unités lexicales sous formes d'expressions rationnelles et d'associer une action à chacune d'elles.

\paragraph{}
On peut donc définir les unités lexicales utilisées dans un programme écrit en SoS. Dans le fichier \textsf{fsos.lex}, on définit tout d'abord les unités lexicales réservées au langage:
\begin{itemize}
    \item Les symboles (\textbf{+}, \textbf{-}, \textbf{*}, \textbf{()}...)
    \item Les mots clés (\textbf{if}, \textbf{for}, \textbf{test}...)
\end{itemize}
Puis les unités lexicales définies par l'utilisateur:
\begin{itemize}
    \item Chaines de caractères
    \item Nombres
    \item Identifiants de variables ou de fonctions
    \item Commentaires
    \item Espaces et tabulations
\end{itemize}

\paragraph{}
On associe ensuite une action à chaque unité lexicale.\\
Pour les unités lexicales réservées au langages, cela consiste à renvoyer un \textit{TOKEN}, c'est à dire une valeur numérique correspondant à une unité lexicale.
\begin{figure}[H]
\begin{lstlisting}
\+ return PLUS;
if return IF;
\end{lstlisting}
\caption{Exemple d'action \textsf{Flex} pour le symbole \textbf{+} et le mot clé \textbf{if}}
\end{figure}

Les unités lexicales ayant une valeur définie par l'utilisateur doivent être ignorées ou transmises (à l'aide de \textsf{yylval}, \textsf{yytext} ainsi qu'un TOKEN).
\begin{figure}[H]
\begin{lstlisting}
#[^\n]*\n ;
(([1-9][0-9]*)|0) {yylval.val = strdup(yytext); return INTEGER;}
\end{lstlisting}
\caption{Exemple d'action \textsf{Flex} pour les commentaires et les entiers}
\end{figure}

\noindent
\begin{minipage}[!hc]{0.12\textwidth}
   \textbf{Remarque}
\end{minipage}
\vrule\enskip\vrule\quad\begin{minipage}{\dimexpr 0.87\textwidth-0.8pt-1.5em}
Les unités lexicales qui ne sont pas reconnues par l'analyseur lexical sont considérées incorrectes pour un programme SoS et mettent fin à la compilation.
\end{minipage}

\paragraph{}
Les valeurs renvoyées par l'analyseur lexical \textsf{Flex} sont transmises à l'analyseur syntaxique.