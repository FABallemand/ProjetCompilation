\section{Analyse Syntaxique}

\paragraph{}
Après avoir déterminé les blocs de taille minimale composant le programme, il faut vérifier s'ils sont assemblés de façon correcte. C'est à dire si le programmeur a écrit des instructions correctes et agencées convenablement selon la grammaire du langage de programmation.

\paragraph{}
L'analyse syntaxique est réalisée à l'aide de \textsf{Bison}, un outil permettant de définir la grammaire d'un langage de programmation et de définir des actions à effectuer pour chaque règle rencontrée dans le programme.

\paragraph{}
L'exemple de règle ci-dessous correspond à la multiplication de deux entiers. On y trouve:
\begin{itemize}
    \item Une instruction qui permet d'afficher la règle lorsqu'elle est utilisée (utilisé pour du debuggage)
    \item La création d'une opérande de type variable utilisable dans une instruction de code intermédiaire
    \item Un appel à la fonction \textsf{genCode} qui permet de générer le code intermédiaire (code à trois adresses) correspondant.
    \item Des affectations de valeurs de l'élément de droite à l'élément de gauche de la règle selon leur type
\end{itemize}
\begin{figure}[H]
\begin{lstlisting}
produit_entier
: produit_entier STAR operande_entier
{
    if(DEBUG)
        printRule("produit_entier STAR operande_entier");
    $$.firstquad = $1.firstquad;
    struct quadop result = quadop_var(newtemp());
    genCode(quad_new(Q_MUL, $1.result, $3.result, result));
    $$.result = result;
}
\end{lstlisting}
\caption{Exemple d'action \textsf{Bison} pour la multiplication de deux entiers}
\end{figure}

\paragraph{}
La première partie du fichier \textsf{bsos.y} permet de définir des propriétés de la grammaire (priorité des opérations) ainsi que des outils pour l'analyse syntaxique (TOKEN et types d'éléments de règles).\\
Les types définis dans la section \textsf{\%union} permettent d'utiliser chaque élément de règle comme une structure pour y stocker des information à porpager lors de la compilation.
\begin{figure}[H]
\begin{lstlisting}
%union{
    struct {
        size_t firstquad;
        struct quadop result;
    } expr_val;
}

%type <expr_val> produit_entier
\end{lstlisting}
\caption{Déclaration du type \textsf{expr\_val} dans le fichier \textsf{Bison}}
\end{figure}
Dans l'exemple précédent on utilise \textsf{dollar dollar} pour accéder aux champs de \textsf{produit\_entier} et y affecter les valeurs de l'addresse du premier \textit{quadruplet} (code à trois adresses) correspondant à la multiplication et le resultat de cette multiplication.\\

\paragraph{}
La partie la plus importante de fichier \textsf{Bison} (la deuxième) contient toutes les règles de la grammaire et leurs actions.\\
\`A cette étape de la compilation, on génère du code intermédiaire grâce à la fonction \textsf{genCode}. Les différents types de quadruplets (\textsf{Q\_ADD}, \textsf{Q\_EQUAL\_STRING}, \textsf{Q\_GOTO}...) et types d'opérandes de quadruplets (\textsf{QO\_CST}, \textsf{QO\_VAR}, \textsf{QO\_UNKNOWN}...) sont définis dans le fichiers \textsf{quad.h}.\\
L'adresse d'une instruction correspond à sa position dans un tableau de quadruplets contenant toutes les instructions du programme. Certaines instructions, notamment les \textsf{Q\_GOTO}, contiennent des opérandes qui ne sont pas connues au moment où elles sont compilées. Il faut donc les enregistrer dans le quadruplet comme \textsf{QO\_UNKNOWN}, les enregistrer dans le membre gauche de la règle dans un liste d'opérandes à compléter (\textsf{next}, \textsf{ltrue} ou \textsf{lfalse}) et les compléter par la suite grâce à la fonction \textsf{complete}.
\paragraph{}
C'est aussi à cette étape de la compilation que la table des symboles est créée. Cette structure a pour but d'enregistrer les variables déclarées dans le programme SoS à compiler ainsi que leur portées. On utilise pour cela une pile de contextes (voir \textsf{symbol\_table.h} et \textsf{symbol\_table.c}). Dans l'exemple ci-dessous, on constate que la fonction \textsf{pushContext} est utilisée pour empiler le contexte global.
\begin{figure}[H]
\begin{lstlisting}
initialisation 
: %empty
{
    if(DEBUG)
        printRule("empty (initialisation)");
    pushContext();
    newName(S_GLOBAL, status, VAR, 0);
    genCode(quad_new(Q_AFFECT, quadop_cst(zero), quadop_empty(), quadop_var(status)));
}
\end{lstlisting}
\caption{Actions réalisées pour initialiser un programme lors de l'analyse syntaxique}
\end{figure}
On remarque aussi dans l'exemple ci-dessus l'appel à la fonction \textsf{newName} qui permet de créer une nouvelle variable dans la table des symboles, ici pour créer la variable \textsf{?} (identifiant invalide pour un utilisateur)qui contient le status (le code de retour du programme). Une stratégie similaire est utilisée pour le code retour d'une fonction.\\
La fonction \textsf{newName} est aussi utilisée pour créer des variables temporaires dans le table des symboles qui ont elles aussi un identifiant invalide pour un utilisateur afin d'éviter tout conflit entre varibles.\\
Par la suite, on peut vérifier si une variable est dans le contexte voulu grâce à la fonction \textsf{lookUp} et en précisant le niveau de contexte souhaité (courrant et/ou global).